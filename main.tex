\documentclass{article}
\usepackage{amsmath}
\usepackage{amsfonts}
\usepackage{amssymb}
\usepackage{ctex}
\usepackage{graphicx}
\usepackage{float}
\usepackage{geometry}
\geometry{a4paper,scale=0.8}
\usepackage{caption}
\usepackage{subcaption}
\newcommand{\oiint}{\mathop{{\int\!\!\!\!\!\int}\mkern-21mu \bigcirc} {}}
\newcommand*{\dif}{\mathop{}\!\mathrm{d}}
\newcommand*{\md}{\mathop{}\!\mathrm{d}}

\usepackage{parskip}
\setlength{\parindent}{0cm}

\usepackage{bm}
\let\Oldmathbf\mathbf
\renewcommand{\mathbf}[1]{\boldsymbol{\Oldmathbf{#1}}}
\let\eqnarray\align

\author{Xiping Hu}
\title{数学物理方法}

\begin{document}
\maketitle

\section{勒让德方程}

\subsection{在$x=0$的邻域求解勒让德方程}

$l$阶勒让德方程
\begin{equation*}
  \label{eq:1}
  \left( 1 - x^{2} \right) y'' - 2 x y' + l \left( l + 1 \right) y = 0
\end{equation*}
的解法为:设
\begin{equation*}
  \label{eq:2}
  y \left( x \right) = \sum\limits_{k = 0}^{\infty} a_{k} z^{k}
\end{equation*}
则
\begin{equation*}
  \begin{aligned}
    y \left( x\right) &= a_{0} + a_{1}x + a_{2}x^{2} + a_{3}x^{3} + \cdots \\
    y' \left( x \right) &= a_{1}x + 2 a_{2}x + 3 a_{3}x^{2} + 4 a_{4}x^{3} \cdots \\
    y'' \left( x \right) &= 2 a_{2} + 3 \cdot 2  a_{3}x + 4 \cdot 3 a_{4}x^{2} + 5 \cdot 4 a_{5} x^{3} \cdots
  \end{aligned}
\end{equaotion*}
代入方程,比较系数,得到奇数项和偶数项

\subsection{勒让德多项式}

\begin{equation*}
  \label{eq:6}
  P_{l} \left( x \right) = \dfrac{1}{2^{l} l!} \dfrac{\md^{l}}{\md x^{l}} \left( x^{2} - 1 \right)^{l}
\end{equation*}

\subsection{广义傅立叶级数的展开}

\begin{equation*}
  \label{eq:7}
  \left\{
    \begin{aligned}
      f(x) &= \sum\limits_{0}^{\infty} f_{l} P_{l}(x) \\
      f_{l} &= \dfrac{2l+1}{2} \int_{-1}^{+1} f(x) P_{l}(x) \md x
    \end{aligned}
  \right.
\end{equation*}
\end{document}