\documentclass{article}
\usepackage{amsmath}
\usepackage{amsfonts}
\usepackage{amssymb}
\usepackage{wasysym}
\usepackage{ctex}
\usepackage{graphicx}
\usepackage{float}
\usepackage{geometry}
\geometry{a4paper,scale=0.8}
\usepackage{caption}
\usepackage{subcaption}
% \newcommand{\oiint}{\mathop{{\int\!\!\!\!\!\int}\mkern-21mu \bigcirc} {}}
\newcommand*{\dif}{\mathop{}\!\mathrm{d}}
\newcommand*{\md}{\mathop{}\!\mathrm{d}}

\usepackage{parskip}
\setlength{\parindent}{0cm}

\usepackage{bm}
\let\Oldmathbf\mathbf
\renewcommand{\mathbf}[1]{\boldsymbol{\Oldmathbf{#1}}}
\let\eqnarray\align

\author{Xiping Hu}
\title{数学物理方法要背的东西汇总}

\begin{document}
\maketitle

\section{解析函数}

函数$f(x) = u + v \mathrm{i}$解析的充分必要条件
\begin{equation*}
  \left\{
  \begin{aligned}
    \dfrac{\partial u}{\partial x} &= + \dfrac{\partial v}{\partial y} \\
    \dfrac{\partial u}{\partial y} &= - \dfrac{\partial v}{\partial x} 
  \end{aligned}
  \right.
\end{equation*}

\section{复变函数的积分}

\subsection{柯西定理}
对于单连通区域
\begin{equation*}
  \oint_{c} f(z) \md z = 0
\end{equation*}

\subsection{柯西公式}

单连通区域内点$z$
\begin{equation*}
  f(z) = \dfrac{1}{2 \mathrm{\pi} \mathrm{i}} \oint_{l} \dfrac{f(\zeta)}{\zeta - z} \md \zeta
\end{equation*}

\section{留数定理}

$b_{i}$为所有$l$内奇点,则
\begin{equation*}
  \oint_{l} f(z) \md z = 2 \mathrm{\pi} \mathrm{i} \sum\limits \mathrm{Res}f(b_{i})
\end{equation*}
包括无穷远点在内的所有的奇点的留数和为零

\section{留数的运算}
$n$阶极点留数运算
\begin{equation*}
  \mathrm{Res} \left[ f(z),\eta \right] = \dfrac{1}{(n - 1)!} \lim\limits_{z \rightarrow \eta} \dfrac{\md^{n-1}}{\md z^{n-1}} \left[ (z-\eta)^{n} f(z) \right]
\end{equation*}
当$f(z) = \dfrac{P(z)}{Q(z)} $ 时,一阶级点的计算可以简化
\begin{equation*}
  \mathrm{Res} f(z_{0}) = \lim\limits_{z \rightarrow z_{0}} (z - z_{0}) f(z) = \lim\limits_{z \rightarrow z_{0}} (z - z_{0}) \dfrac{P(z)}{Q(z)} =  \dfrac{P(z_{0})}{Q'(z_{0})} 
\end{equation*}

\section{利用留数定理计算实变函数的积分}

\subsection{几个类型的积分}



\emph{非常重要,请仔细阅读梁昆淼《数学物理方法》P56-P63,就那几种类型}

\subsection{大圆弧定理}

当$z \rightarrow \infty$时,如果$zf(z) \rightarrow k$,积分路径是半径为$R$的圆弧$C_{R}$,则
\begin{equation*}
  \lim\limits_{R \rightarrow \infty} \int_{C_{R}} f(z) \md z = \mathrm{i} k (\beta - \alpha)
\end{equation*}

\subsection{小圆弧定理}


当$z \rightarrow a$时,如果$(z-a)f(z) \rightarrow k$,积分路径是半径为$R$的圆弧$C_{R}$,则
\begin{equation*}
  \lim\limits_{\rho \rightarrow 0} \int_{C_{\rho}} f(z) \md z = \mathrm{i} k \left( \beta - \alpha  \right)
\end{equation*}

\section{波动方程的行波解}

达朗贝尔解
\begin{equation*}
  u(x,t) = \dfrac{1}{2} \left[ \phi (x + vt) + \phi (x - vt) \right] + \dfrac{1}{2v} \int_{x-vt}^{x+vt} \psi(\xi) \md \xi
\end{equation*}

\section{数学物理方程的求解}

\subsection{齐次边界条件}

\subsubsection{波动方程}


\begin{equation*}
  \left\{
  \begin{aligned}
    & \dfrac{\partial^{2} u}{\partial t^{2}} - v^{2} \dfrac{\partial^{2} u}{\partial x^{2}} = 0 \\
    & u |_{x=0} = 0\\
    & u |_{x=L} = 0\\
    & u |_{t=0} = \phi (x)\\
    & \!\!\left. \dfrac{\partial u}{\partial t} \right|_{t=0} = \psi (x)
  \end{aligned}
  \right.
\end{equation*}
将分离变量形式的解
\begin{equation*}
  u(x,t) = X(x) T(t)
\end{equation*}
带入方程,求解本征值
\begin{equation*}
  \dfrac{1}{v^{2}} \dfrac{T''(t)}{T(t)} = \dfrac{X''(x)}{X(x)} = -\lambda
\end{equation*}
\begin{equation*}
  \begin{aligned}
    T''(t) + \lambda v^{2} T(t) &= 0\\
    X''(x) + \lambda X(x) &= 0
  \end{aligned}
\end{equation*}
设$X(x)$的解$X(x) = A \sin \sqrt{\lambda} x + B \cos \sqrt{\lambda}x$,
代入边界条件$u|_{x=0}=0$和$u|_{x=L} = 0$
\begin{equation*}
  \begin{aligned}
    B &= 0\\
    A \sin \sqrt{\lambda}x &= 0
  \end{aligned}
\end{equation*}
本征值与本征函数
\begin{equation*}
  \begin{aligned}
    \lambda_{n} &= \left( \dfrac{n \mathrm{\pi}}{l}  \right)^{2}\\
    X_{n}(x) &=
  \end{aligned}
\end{equation*}
用本征函数解$T(t)$
\begin{equation*}
  \begin{aligned}
    T_{n}(t) = C_{n} \sin \dfrac{n \mathrm{\pi} v}{l}t + D_{n} \cos \dfrac{n \mathrm{\pi} v}{l}t
  \end{aligned}
\end{equation*}
叠加出$u(x,t)$
\begin{equation*}
  u(x,t) = \sum_{n=1}^{\infty} \left( C_{n} \sin \dfrac{n \mathrm{\pi} v}{l}t + D_{n} \cos \dfrac{n \mathrm{\pi} v}{l}t \right)  \sin \dfrac{n \mathrm{\pi}}{l}x 
\end{equation*}
代入
\begin{equation*}
  \left\{
  \begin{aligned}
    & u |_{t=0} = \phi (x)\\
    & \!\!\left. \dfrac{\partial u}{\partial t} \right|_{t=0} = \psi (x)
  \end{aligned}
  \right.
\end{equation*}
利用傅立叶积分计算$C_{n}$和$D_{n}$

\subsubsection{矩形区域的稳定问题}

\begin{equation*}
  \left\{
  \begin{aligned}
    & \dfrac{\partial^{2} u}{\partial x^{2}} + \dfrac{\partial^{2} u}{\partial y^{2}} = 0\\
    & u|_{x=0} = 0\\
    & u|_{y=0} = f(x)\\
    & \!\!\left. \dfrac{\partial u}{\partial x} \right|_{x=a} = 0\\
    & \!\!\left. \dfrac{\partial u }{\partial y} \right|_{y=b} = 0
  \end{aligned}
  \right.
\end{equation*}
设
\begin{equation*}
  u \left( x,y \right) = X(x)Y(y)
\end{equation*}
则
\begin{equation*}
  \dfrac{X''(x)}{X(x)} = - \dfrac{Y''(y)}{Y(y)} = - \lambda 
\end{equation*}

求解本征值问题
\begin{equation*}
  \begin{aligned}
    X''(x) + \lambda X(x) &= 0\\
    Y''(y) - \lambda Y(y) &= 0
  \end{aligned}
\end{equation*}
边界条件给出
\begin{equation*}
  \begin{aligned}
    X(0) &=  0\\
    X'(a) &= 0
  \end{aligned}
\end{equation*}
设
\begin{equation*}
  X(x) = A \sin \sqrt{\lambda} x + B \cos \sqrt{\lambda} x
\end{equation*}
则
\begin{equation*}
  \begin{aligned}
    B &= 0\\
    \cos \sqrt{\lambda} a &= 0
  \end{aligned}
\end{equation*}
本征值
\begin{equation*}
  \sqrt{\lambda_{n}} a = \dfrac{2n+1}{2} \mathrm{\pi}
\end{equation*}
\begin{equation*}
  \lambda_{n} = \left( \dfrac{2n+1}{2a} \mathrm{\pi} \right)^{2}
\end{equation*}
对应本正函数
\begin{equation*}
  X(x) = \sin  \dfrac{2n+1}{2a} \mathrm{\pi}  x
\end{equation*}
$Y(y)$的解为
\begin{equation*}
  Y(y) = C \cosh \sqrt{\lambda}x + D \sinh \sqrt{\lambda} x
\end{equation*}
代入本征值
\begin{equation*}
  Y_{n}(y) = C_{n} \cosh \dfrac{2n+1}{2a} \mathrm{\pi} x + D_{n} \sinh \dfrac{2n+1}{2a} \mathrm{\pi} x
\end{equation*}
分离变量形式的解
\begin{equation*}
  u(x,y) = \sum\limits_{n=0}^{\infty} \left( C_{n} \cosh \dfrac{2n+1}{2a} \mathrm{\pi} x + D_{n} \sinh \dfrac{2n+1}{2a} \mathrm{\pi} x \right)\sin  \dfrac{2n+1}{2a} \mathrm{\pi}  x
\end{equation*}
代入边界条件,使用傅立叶积分计算常数


\subsection{非其次方程(特解法)}

\begin{equation*}
  \left\{
  \begin{aligned}
    & \dfrac{\partial^{2} u}{\partial t^{2}} - a^{2} \dfrac{\partial^{2} u}{\partial x^{2}} = f(x,t) \\
    & u |_{x=0} = 0\\
    & u |_{x=L} = 0\\
    & u |_{t=0} = 0\\
    & \!\!\left. \dfrac{\partial u}{\partial t} \right|_{t=0} = 0
  \end{aligned}
  \right.
\end{equation*}
设
\begin{equation*}
  u(x,t) = v(x,t) + w(x,t)
\end{equation*}
其中$v(x,t)$可以不满足初始条件
\begin{equation*}
  \left\{
  \begin{aligned}
    & \dfrac{\partial^{2} v}{\partial t^{2}} - a^{2} \dfrac{\partial^{2} v}{\partial x^{2}} = f(x,t) \\
    & v |_{x=0} = 0\\
    & v |_{x=L} = 0
  \end{aligned}
  \right.
\end{equation*}

$w(x,t)$满足
\begin{equation*}
  \left\{
  \begin{aligned}
    & \dfrac{\partial^{2} w}{\partial t^{2}} - a^{2} \dfrac{\partial^{2} w}{\partial x^{2}} = 0 \\
    & w |_{x=0} =  -v|_{x=0} = 0\\
    & w |_{x=L} = -v|_{x=L} = 0\\
    & w |_{t=0} = -v|_{t=0}\\
    & \!\!\left. \dfrac{\partial w}{\partial t} \right|_{t=0} = - \left. \dfrac{\partial v}{\partial t} \right|_{t=0}
  \end{aligned}
  \right.
\end{equation*}

\subsection{非齐次方程(展开法)}
\begin{equation*}
  \left\{
  \begin{aligned}
    & \dfrac{\partial^{2} u}{\partial t^{2}} - a^{2} \dfrac{\partial^{2} u}{\partial x^{2}} = f(x,t) \\
    & u |_{x=0} = 0\\
    & u |_{x=l} = 0\\
    & u |_{t=0} = 0\\
    & \!\!\left. \dfrac{\partial u}{\partial t} \right|_{t=0} = 0
  \end{aligned}
  \right.
\end{equation*}

将$f(x,t)$用相应齐次方程的本征函数$X_{n}(x)$展开
\begin{equation*}
  f(x,t) = \sum\limits_{n=1}^{\infty} g_{n}(t) X_{n}(x)
\end{equation*}
于是有
\begin{equation*}
  \sum\limits_{n=1}^{\infty} X_{n}(x)T_{n}''(t) - a^{2} X_{n}''(x)T_{n}(t) = \sum\limits_{n=1}^{\infty} g_{n}(t) X_{n}(x)
\end{equation*}
由于
\begin{equation*}
  X_{n}''(x) + \lambda_{n} X(x) = 0
\end{equation*}
有
\begin{equation*}
  \sum\limits_{n=1}^{\infty} X_{n}(x)T_{n}''(t) + a^{2} \lambda_{n} X_{n}(x)T_{n}(t) = \sum\limits_{n=1}^{\infty} g_{n}(t) X_{n}(x)
\end{equation*}
即
\begin{equation*}
  T_{n}''(t) + a^{2} \lambda_{n}T_{n}(t) = g_{n}(t)
\end{equation*}
初始条件给出
\begin{equation*}
  \begin{aligned}
    \sum\limits_{n=1}^{\infty} X_{n}(0)T_{n}(0) &= 0\\
    \sum\limits_{n=1}^{\infty} X_{n}(0)T_{n}'(0) &= 0
  \end{aligned}
\end{equation*}
即
\begin{equation*}
  \begin{aligned}
    T_{n}(0) &= 0\\
    T_{n}'(0) &= 0
  \end{aligned}
\end{equation*}
解为
\begin{equation*}
  T_{n}(t) = \dfrac{l}{n \mathrm{\pi} a} \int_{0}^{\tau} g_{n} (\tau) \sin \dfrac{n \mathrm{\pi} a}{l} (t - \tau ) \md \tau 
\end{equation*}

\section{傅立叶级数展开}

\subsection{奇函数,周期$2l$}

\begin{equation*}
  f(x) = \sum\limits_{i=1}^{n}b_{k} \sin \dfrac{k \mathrm{\pi} x}{l} 
\end{equation*}
\begin{equation*}
  b_{k} = \dfrac{2}{l} \int_{0}^{l} f(\xi) \sin \dfrac{k \mathrm{\pi} \xi}{l} \md \xi 
\end{equation*}
\subsection{偶函数,周期$2l$}

\begin{equation*}
  f(x) = \dfrac{a_{0}}{2} + \sum\limits_{i=1}^{n}a_{k} \cos \dfrac{k \mathrm{\pi} x}{l} 
\end{equation*}
\begin{equation*}
  a_{k} = \dfrac{2}{l} \int_{0}^{l} f(\xi) \cos \dfrac{k \mathrm{\pi} \xi}{l} \md \xi 
\end{equation*}


\section{勒让德方程}

\subsection{在$x=0$的邻域求解勒让德方程}

$l$阶勒让德方程
\begin{equation*}
  \label{eq:1}
  \left( 1 - x^{2} \right) y'' - 2 x y' + l \left( l + 1 \right) y = 0
\end{equation*}
的解法为:设
\begin{equation*}
  \label{eq:2}
  y \left( x \right) = \sum\limits_{k = 0}^{\infty} a_{k} z^{k}
\end{equation*}
则
\begin{equation*}
  \begin{aligned}
    y \left( x\right) &= a_{0} + a_{1}x + a_{2}x^{2} + a_{3}x^{3} + \cdots \\
    y' \left( x \right) &= a_{1}x + 2 a_{2}x + 3 a_{3}x^{2} + 4 a_{4}x^{3} \cdots \\
    y'' \left( x \right) &= 2 a_{2} + 3 \cdot 2  a_{3}x + 4 \cdot 3 a_{4}x^{2} + 5 \cdot 4 a_{5} x^{3} \cdots
  \end{aligned}
\end{equation*}
代入方程,比较系数,得到奇数项和偶数项


\subsection{勒让德多项式}

\begin{equation*}
  \label{eq:6}
  P_{l} \left( x \right) = \dfrac{1}{2^{l} l!} \dfrac{\md^{l}}{\md x^{l}} \left( x^{2} - 1 \right)^{l}
\end{equation*}

\subsection{广义傅立叶级数的展开}

\begin{equation*}
  \label{eq:7}
  \left\{
    \begin{aligned}
      f(x) &= \sum\limits_{0}^{\infty} f_{l} P_{l}(x) \\
      f_{l} &= \dfrac{2l+1}{2} \int_{-1}^{+1} f(x) P_{l}(x) \md x
    \end{aligned}
  \right.
\end{equation*}x
\end{document}